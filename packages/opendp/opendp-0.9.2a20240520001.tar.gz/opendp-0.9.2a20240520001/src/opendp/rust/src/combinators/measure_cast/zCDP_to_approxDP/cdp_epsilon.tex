\documentclass{article}
% common styling and macros shared by all proof files

\usepackage[top=1in, right=1in, left=1in, bottom=1.5in]{geometry}

\usepackage{amsmath,amsthm,amsfonts,amssymb,amscd}
\usepackage{listings}
\usepackage{hyperref}
\usepackage{xcolor}
\usepackage{xr}

\usepackage{enumerate} 
\usepackage{physics}
\usepackage{fancyhdr}
\usepackage{hyperref}
\usepackage{graphicx}
\usepackage{tcolorbox}
\usepackage{catchfile}
\usepackage{pdftexcmds}
\usepackage[T1]{fontenc}

% hyperref
\hypersetup{
  colorlinks=true,
  linkcolor=blue,
  linkbordercolor={0 0 1}
}

% \contrib macro to indicate inclusion in "contrib".
\usepackage{tcolorbox}
\newtcolorbox{warn_box}{colback=red!5!white,colframe=red!75!black}
\newcommand{\contrib}{{\begin{warn_box}This proof resides in \textbf{``contrib''} because it has not completed the vetting process.\end{warn_box}}} 
\newcommand{\floatingPoint}{{\begin{warn_box}This implementation is susceptible to floating-point vulnerabilities.\end{warn_box}}} 

% asOfCommit macro to version a code dependency. Arguments:
%    #1: relative path to file you are dependent on
%    #2: commit hash it was last edited. If outdated, this should be the second hash in the footnoote. Otherwise,
%            git log -n 1 --pretty=format:%h -- path/to/file.rs
\makeatletter
\ifnum\pdf@shellescape=1
   % "private" command that builds a link to a blob
  \newcommand{\linkOpendpBlob}[3]{%
    \href{https://github.com/opendp/opendp/blob/#1/#2#3}{\path{#3} at commit #1}}

  % latex macro expansion has a separate phase for \input evaluation
  %     immediately evaluate a command to write a temp file to ./out containing the current directory
  \immediate\write18{[ ! -f out/cwd.txt ] && (mkdir -p out && git rev-parse --show-prefix | sed "s|_|\@backslashchar\@backslashchar\@backslashchar_|g" > out/cwd.txt)}
  %     ...and then retrieve the current working directory by loading the temp file
  \CatchFileDef\GitWorkingDir{out/cwd.txt}{\endlinechar=-1}

  % command for building the (up to date) or (outdated) status
  \newcommand{\fileStatus}[2]{%
  \setbox0=\hbox{\input|"git --no-pager log -n1 --pretty='\@percentchar H' #1 | grep -E '^#2.*'"\unskip}\ifdim\wd0=0pt
        (outdated\footnote{See new changes with \texttt{git diff #2..\input|"git --no-pager log -n1 --pretty='\@percentchar h' #1" \GitWorkingDir\path{#1}}})\else
        (up to date)\fi
  }

  \newcommand{\asOfCommit}[2]{%
      % permalink the target
      \linkOpendpBlob{#2}{\GitWorkingDir}{#1}
      % conditionally add (outdated) or (up to date) depending on matching commit hash
      \fileStatus{#1}{#2}%
  }
\else
  % simplified command if shell-escape not enabled
  \newcommand{\asOfCommit}[2]{#1 at commit #2 (unknown status\footnote{Shell-escape is not enabled. Enable \texttt{--shell-escape} to check if this proof is up-to-date with the code.})}
\fi
\makeatother

% \vettingPR macro to link a PR. Arguments:
%    #1: PR number
\newcommand{\vettingPR}[1]{\href{https://github.com/opendp/opendp/pull/#1}{Pull Request \##1}}

% for links to rustdoc items in OpenDP. Arguments:
%    #1: path to item, and designation as trait, struct, fn, etc.
%    #2: item name
\makeatletter
\ifnum\pdf@shellescape=1
  % latex macro expansion has a separate phase for \input evaluation
  %     immediately evaluate a command to write a temp file to ./out containing the base path
  \immediate\write18{[ ! -f out/rustdoc.txt ] && mkdir -p out && ([ -z `kpsewhich --var-value OPENDP_RUSTDOC_PORT` ] && echo "https://docs.rs/opendp/`head -n 1 \@backslashchar`git rev-parse --show-toplevel\@backslashchar`/VERSION | sed 's|.*-dev.*|latest|g'`" || echo "http://localhost:`kpsewhich --var-value OPENDP_RUSTDOC_PORT`") > out/rustdoc.txt}
  %     ...and then retrieve the base path by loading the temp file
  \CatchFileDef\OpenDPRustdocBase{out/rustdoc.txt}{\endlinechar=-1}
\else
  % if shell commands are not enabled, just claim latest
  \newcommand{\OpenDPRustdocBase}{https://docs.rs/opendp/latest}
\fi
\makeatother
\newcommand{\rustdoc}[2]{\href{\OpenDPRustdocBase/opendp/#1.#2.html}{\texttt{#2}}}

% for links to external dependencies. Arguments:
%    #1: crate name
%    #2: path to item, and designation as trait, struct, fn, etc.
%    #3: item name
\newcommand{\docsrs}[3]{\href{https://docs.rs/#1/latest/#1/#2.#3.html}{\texttt{#3}}}

% minted (pseudocode)
\definecolor{codegreen}{rgb}{0,0.6,0}
\definecolor{codegray}{rgb}{0.5,0.5,0.5}
\definecolor{codepurple}{rgb}{0.58,0,0.82}
\definecolor{backcolour}{rgb}{0.95,0.95,0.92}

\lstdefinestyle{mystyle}{
    backgroundcolor=\color{backcolour},   
    commentstyle=\color{codegreen},
    keywordstyle=\color{magenta},
    numberstyle=\tiny\color{codegray},
    stringstyle=\color{codepurple},
    basicstyle=\ttfamily\footnotesize,
    breakatwhitespace=false,         
    breaklines=true,                 
    captionpos=b,                    
    keepspaces=true,                 
    numbers=left,                    
    numbersep=5pt,                  
    showspaces=false,                
    showstringspaces=false,
    showtabs=false,                  
    tabsize=2
}

\lstset{style=mystyle}

% common commands
\theoremstyle{definition}
\newtheorem{theorem}{Theorem}[section]
\newtheorem{lemma}[theorem]{Lemma}
\newtheorem{definition}[theorem]{Definition}
\newtheorem{warning}{Warning}
\newtheorem{corollary}{Corollary}
\newtheorem{proposition}{Proposition}
\newtheorem{remark}{Remark}
\newtheorem{observation}{Observation}
\newtheorem{note}{Note}

\newcommand{\vicki}[1]{{ {\color{olive}{(vicki)~#1}}}}
\newcommand{\hanwen}[1]{{ {\color{purple}{(hanwen)~#1}}}}
\newcommand{\zach}[1]{{ {\color{red}{(zach)~#1}}}}

\newcommand{\MultiSet}{\mathrm{MultiSet}}
\newcommand{\len}{\mathrm{len}}
\newcommand{\din}{\texttt{d\_in}}
\newcommand{\dout}{\texttt{d\_out}}
\newcommand{\T}{\texttt{T} }
\newcommand{\F}{\texttt{F} }
\newcommand{\Map}{\texttt{Map}}
\newcommand{\X}{\mathcal{X}}
\newcommand{\Y}{\mathcal{Y}}
\newcommand{\True}{\texttt{True}}
\newcommand{\False}{\texttt{False}}
\newcommand{\clamp}{\texttt{clamp}}
\newcommand{\function}{\texttt{function}}
\newcommand{\float}{\texttt{float }}
\newcommand{\questionc}[1]{\textcolor{red}{\textbf{Question:} #1}}


\newcommand{\validTransformation}[2]{%
  For every setting of the input parameters #1 to #2 such that the given preconditions
  hold, #2 raises an exception (at compile time or run time) or returns a valid transformation. A valid transformation has the following properties:
  \begin{enumerate}
      \item \textup{(Appropriate output domain).} 
      For every element $v$ in \texttt{input\_domain}, $\function(v)$ is in \texttt{output\_domain} or raises a data-independent runtime exception.
      
      \item \textup{(Stability guarantee).} 
      For every pair of elements $u, v$ in \texttt{input\_domain} and for every pair $(\din, \dout)$, 
      where \din\ has the associated type for \texttt{input\_metric} and \dout\ has the associated type for \\ \texttt{output\_metric}, 
      if $u, v$ are \din-close under \texttt{input\_metric}, $\texttt{stability\_map}(\din)$ does not raise an exception,
      and $\texttt{stability\_map}(\din) \leq \dout$, 
      then $\function(u), \function(v)$ are $\dout$-close under \texttt{output\_metric}.
  \end{enumerate}
}


\newcommand{\validMeasurement}[2]{%
  For every setting of the input parameters #1 to #2 such that the given preconditions
  hold, #2 raises an exception (at compile time or run time) or returns a valid measurement. A valid measurement has the following property:
  \begin{enumerate}
      \item \textup{(Privacy guarantee).}
      For every pair of elements $u, v$ in \texttt{input\_domain} and for every pair $(\din, \dout)$,
      where \din\ has the associated type for \texttt{input\_metric} and \dout\ has the associated type for \\ \texttt{output\_measure},
      if $u, v$ are \din-close under \texttt{input\_metric}, $\texttt{privacy\_map}(\din)$ does not raise an exception,
      and $\texttt{privacy\_map}(\din) \leq \dout$,
      then $\function(u), \function(v)$ are $\dout$-close under \texttt{output\_measure}.
  \end{enumerate}
}



\title{\texttt{fn cdp\_epsilon}}
\author{Michael Shoemate}

\begin{document}
\maketitle

Proves soundness of \texttt{fn cdp\_epsilon} in \asOfCommit{cdp_epsilon.rs}{0b8f4222}.
This proof is an adaptation of \href{https://arxiv.org/pdf/2004.00010.pdf#subsection.2.3}{subsection 2.3} of \cite{CKS20}.
\section{Bound Derivation}

\begin{definition}
\label{plrv}
(Privacy Loss Random Variable). 
Let $M: \mathcal{X}^n \rightarrow \mathbb{Y}$ be a randomized algorithm.
Let $x, x' \in \mathcal{X}^n$ be neighboring inputs.
Define $f: \mathcal{Y} \rightarrow \mathbb{R}$ by $f(y) = log \left( \frac{\mathbb{P}[M(x)=y]}{\mathbb{P}[M(x')=y]} \right)$.
Let $Z = f(M(x))$, the privacy loss random variable, denoted $Z \leftarrow PrivLoss(M(x) || M(x'))$.
\end{definition}

\begin{lemma}
\label{delta-bound}
\cite{CKS20} Let $\epsilon, \delta \geq 0$. Let M: $\mathcal{X}^n \rightarrow \mathcal{Y}$ be a randomized algorithm. Then M satisfies $(\epsilon, \delta)$-differential privacy if and only if

\begin{align}
    \delta &\geq \underset{Z \leftarrow PrivLoss(M(x)|| M(x'))}{\mathbb{E}} [max(0, 1 - e^{\epsilon - Z})] \\
    % &= \underset{Z \leftarrow PrivLoss(M(x)|| M(x'))}{\mathbb{P}} [Z > \epsilon] - e^\epsilon \underset{Z' \leftarrow PrivLoss(M(x')|| M(x))}{\mathbb{P}} [-Z' > \epsilon] \\
    % &= \int_\epsilon^\infty e^{\epsilon - z} \underset{Z \leftarrow PrivLoss(M(x)|| M(x'))}{\mathbb{P}} [Z > z]dz
\end{align}
for all $x, x' \in \mathcal{X}^n$ differing on a single element.

\begin{proof}
Fix neighboring inputs $x, x' \in \mathcal{X}^n$. 
Let $f: \mathcal{Y} \rightarrow \mathbb{R}$ be as in \ref{plrv}.
For notational simplicity, let $Y = M(x)$, $Y' = M(x')$, $Z = f(Y)$ and $Z' = -f(Y')$.
This is equivalent to $Z \leftarrow PrivLoss(M(x) || M(x'))$.
Our first goal is to prove that

\begin{equation}
    \sup\limits_{E \subset \mathcal{Y}} \mathbb{P}[Y \in E] - e^\epsilon \mathbb{P}[Y' \in E] = \mathbb{E}[max\{0, 1 - e^{\epsilon - Z}\}].
\end{equation}

For any $E \subset \mathcal{Y}$, we have
\begin{equation}
    \mathbb{P}[Y' \in E] = \mathbb{E}[\mathbb{I}[Y' \in E]] = \mathbb{E}[\mathbb{I}[Y \in E] e^{-f(Y)}].
\end{equation}
This is because $e^{-f(y)} = \frac{\mathbb{P}[Y=y]}{\mathbb{P}[Y'=y]}$.

Thus, for all $E \subset \mathcal{Y}$, we have 
\begin{equation}
    \mathbb{P}[Y \in E] - e^\epsilon \mathbb{P}[Y' \in E] = \mathbb{E} \left[ \mathbb{I}[Y \in E] (1 - e^{\epsilon - f(Y)}) \right]
\end{equation}

Now it is easy to identify the worst event as $E = \{y \in \mathcal{Y} : 1 - e^{\epsilon - f(y)} > 0\}$. Thus
\begin{equation}
    \sup\limits_{E \subset Y} \mathbb{P}[Y \in E] - e^{\epsilon} \mathbb{P}[Y' \in E] = \mathbb{E} \left[ \mathbb{I}[1 - e^{\epsilon - f(Y)} > 0] (1 - e^{\epsilon - f(Y)}) \right] = \mathbb{E}[max\{ 0, 1 - e^{\epsilon - Z} \}]
\end{equation}

% Alternatively, since the worst event is equivalently $E = \{y \in \mathcal{Y} : f(y) > \epsilon \}$, we have

% \begin{equation}
%     \sup\limits_{E \subset Y} \mathbb{P}[Y \in E] - e^{\epsilon} \mathbb{P}[Y' \in E] =
%     \mathbb{P} [f(Y) > \epsilon] - e^\epsilon \mathbb{P} [f(Y') > \epsilon] =
%     \mathbb{P} [Z > \epsilon] - e^\epsilon \mathbb{P}[-Z' > \epsilon]
% \end{equation}

% It only remains to show that
% \begin{equation}
%     \mathbb{E} [max \{0, 1 - e^{\epsilon - Z}\}] = \int_\epsilon^\infty e^{\epsilon - z} \mathbb{P}[Z > z] dz.
% \end{equation}

% This follows from integration by parts: Let $u(z) = \mathbb{P}[Z > z]$ and $v(z) = 1 - e^{\epsilon - z}$ and $w(z) = u(z)v(z)$. Then
% \begin{align}
%     \mathbb{E}[max\{ 0, 1 - e^{\epsilon - Z} \}] &= \int_\epsilon^\infty v(z) u'(z) dz = \int_\epsilon^\infty (w'(z) - v'(z) u(z)) dz
%     &= \lim\limits_{z \rightarrow \infty} w(z) - w(\epsilon) + \int_\epsilon^\infty e^{\epsilon - z} \mathbb{P} [Z > z] dz
% \end{align}

% Now $w(\epsilon) = u(\epsilon) (1 - e^{\epsilon - \epsilon} = 0$ and $0 \leq \lim_{z \rightarrow \infty} w(z) \leq \lim_{z \rightarrow \infty} \mathbb{P}[Z > z] = 0$, as required.

\end{proof}

\end{lemma}

\begin{theorem}
\label{renyidp-approxdp-delta}
\cite{CKS20} Let $M: \mathcal{X}^n \rightarrow \mathcal{Y}$ be a randomized algorithm. Let $\alpha \in (1, \infty)$ and $\epsilon \geq 0$. Suppose $D_\alpha(M(x) || M(x')) \leq \tau$ for all $x, x' \in \mathcal{X}^n$ differing in a single entry.\footnote{This is the definition of $(\alpha, \tau)$-R\'enyi differential privacy.} Then M is $(\epsilon, \delta)$-differentially private for 

\begin{equation}
    \delta = \frac{e^{(\alpha - 1) (\tau - \epsilon)}}{\alpha - 1} \left(1 - \frac{1}{\alpha}\right)^\alpha
\end{equation}
\end{theorem}

\begin{proof}
Fix neighboring $x, x' \in  \mathcal{X}^n$ and let $Z \leftarrow PrivLoss(M(x) || M(x'))$. We have
\begin{equation}
    \mathbb{E}[e^{(\alpha - 1) Z}] = e^{(\alpha - 1)D_\alpha(M(x) || M(x'))} \leq e^{(\alpha - 1)\tau}
\end{equation}

By \ref{delta-bound}, our goal is to prove that $\delta \geq \mathbb{E}[max\{0, 1 - e^{\epsilon - Z} \}]$. 
Our approach is to pick $c > 0$ such that $max\{0, 1 - e^{\epsilon - Z} \} \leq c e^{(\alpha - 1) z}$ for all $z \in \mathbb{R}$. Then
\begin{equation}
    \mathbb{E}[max\{0, 1 - e^{\epsilon - Z} \}] \leq \mathbb{E}[c e^{(\alpha - 1) z}] \leq c e^{(\alpha - 1) \tau}.
\end{equation}

We identify the smallest possible value of $c$:
\begin{equation}
    c = \sup\limits_{z \in \mathbb{R}} \frac{max\{0, 1 - e^{\epsilon - z} \}}{e^{(\alpha -1)z}} 
    = \sup\limits_{z \in \mathbb{R}} e^{z - \alpha z} - e^{\epsilon - \alpha z} 
    = \sup\limits_{z \in \mathbb{R}}f(z)
\end{equation}
where $f(z) = e^{z - \alpha z} - e^{\epsilon - \alpha z}$. We have 

\begin{equation}
    f'(z) = e^{z - \alpha z} (1 - \alpha) - e^{\epsilon - \alpha z}(-\alpha) 
    = e^{-\alpha z} (\alpha e^\epsilon - (\alpha - 1) e^z)
\end{equation}

Clearly $f'(z) = 0 \Longleftrightarrow e^z = \frac{\alpha}{\alpha - 1}e^{\epsilon} \Longleftrightarrow z = \epsilon - log(1 - 1/\alpha)$. Thus
\begin{align}
    c &= f(\epsilon - log(1 - 1 / \alpha)) \\
    &= \left(\frac{\alpha}{\alpha - 1} e^\epsilon \right)^{1 - \alpha} - e^{\epsilon} \left(\frac{\alpha}{\alpha - 1} e^\epsilon \right)^{- \alpha} \\
    &= \left(\frac{\alpha}{\alpha - 1} e^\epsilon  - e^\epsilon \right)  \left(\frac{\alpha}{\alpha - 1} e^{-\epsilon} \right)^{\alpha} \\
    &= \frac{e^\epsilon}{\alpha - 1} \left(1 - \frac{1}{\alpha} \right)^\alpha  e^{-\alpha \epsilon}.
\end{align}

Thus
\begin{equation}
    \mathbb{E}[max\{0, 1 - e^{\epsilon - Z}\}] \leq \frac{e^\epsilon}{\alpha - 1}\left(1  - \frac{1}{\alpha} \right) ^\alpha e^{-\alpha \epsilon} e^{(\alpha -1)\tau}
    = \frac{e^{(\alpha - 1)(\tau - \epsilon)}}{\alpha - 1} \left( 1 - \frac{1}{\alpha} \right)^\alpha
    = \delta
\end{equation}

\end{proof}

\begin{corollary}
\label{renyidp-approxdp-epsilon}
\cite{CKS20} Let $M: \mathcal{X}^n \rightarrow \mathcal{Y}$ be a randomized algorithm. Let $\alpha \in (1, \infty)$ and $\epsilon \geq 0$. Suppose $D_\alpha(M(x) || M(x')) \leq \tau$ for all $x, x' \in \mathcal{X}^n$ differing in a single entry. Then M is $(\epsilon, \delta)$-differentially private for 

\begin{equation}
    \epsilon = \tau + \frac{\ln(1 / \delta) + (\alpha - 1) \ln(1 - 1/\alpha) - \ln(\alpha)}{\alpha - 1}
\end{equation}
\end{corollary}

\begin{proof}
This follows by rearranging \ref{renyidp-approxdp-delta}.
\end{proof}

\begin{corollary}
\label{zcdp-approxdp-epsilon}
Let $M: \mathcal{X}^n \rightarrow \mathcal{Y}$ be a randomized algorithm satisfying $\rho$-concentrated differential privacy. Then M is $(\epsilon, \delta)$-differentially private for any $0 < \delta \leq 1$ and 
\begin{equation}
    \epsilon = \inf\limits_{\alpha \in (1, \infty)} \alpha \rho + \frac{\ln(1 / \delta) + (\alpha - 1) \ln(1 - 1/\alpha) - \ln(\alpha)}{\alpha - 1}
\end{equation}
\end{corollary}

\begin{proof}
This follows from \ref{renyidp-approxdp-epsilon} by taking the infimum over all divergence parameters $\alpha$.
\end{proof}

\subsection[Efficient computation of epsilon]{Efficient computation of $\epsilon$}

From \ref{zcdp-approxdp-epsilon}, we have 
\begin{align}
    \epsilon(\alpha) &= \alpha \rho + \frac{\ln(1 / \delta) + (\alpha - 1) \ln((\alpha-1)/\alpha) - \ln(\alpha)}{\alpha - 1} \\
    \epsilon'(\alpha) &= \rho + \frac{\ln(\alpha\delta)}{(\alpha - 1)^2} \\
    \epsilon''(\alpha) &= \frac{2\alpha \ln(\alpha \delta) - \alpha + 1}{(\alpha -1)^3 \alpha}
\end{align}

Notice the curve is convex so long as 
\begin{equation}
    \delta < e^{1/2 - 1/(2\alpha)}/\alpha
\end{equation}
Otherwise the curve is concave, with a non-negative derivative for any choice of $\alpha$: 

\begin{equation}
    \epsilon'(\alpha) = \rho + \frac{\ln(\alpha\delta)}{(\alpha - 1)^2} 
    \geq \rho + \frac{\alpha - 1}{2\alpha (\alpha + 1)^2} 
    \geq 0
\end{equation}

We can find the minimizer $\alpha_{*}$ by conducting a binary search over the interval $(1, \alpha_{max})$, where $\alpha_{max}$ is discovered via exponential search for a positive derivative.


\section{Pseudocode}
\subsubsection*{Precondition}

\begin{itemize}
    \item Type \texttt{Q} must have trait \texttt{Float}.
\end{itemize}

\subsubsection*{Implementation}        
\lstinputlisting[language=Python, firstline=2]{cdp_epsilon.py}

\subsubsection*{Postcondition}
Either a valid epsilon is returned or an error is returned.

\section{Proof}

\begin{theorem}
For any possible setting of $\rho$ and $\delta$, $\texttt{cdp\_epsilon}$ either returns an error, or an $\epsilon$ such that any $\rho$-differentially private measurement is also $(\epsilon, \delta)$-differentially private.
\end{theorem}

\begin{proof}
The code always finds an $\alpha_{*} \approx \texttt{a\_max} \geq 1.01$.
Since $\texttt{a\_max} \in (1, \infty)$, then by \ref{zcdp-approxdp-epsilon}, any $\rho$-differentially private measurement is also $(\epsilon(\texttt{a\_max}), \delta)$-differentially private.
Define $\epsilon_{cons}(\alpha)$ as a ``conservative'' function for computing $\epsilon(\alpha)$, 
where floating-point arithmetic is computed with conservative rounding such that $\epsilon_{cons}(\alpha) \geq \epsilon(\alpha)$ for $\forall \alpha \in (1, \infty)$.
Since $\texttt{epsilon} = \epsilon_{cons}(\texttt{a\_max}) \geq \epsilon(\texttt{a\_max})$, then any $(\epsilon(\texttt{a\_max}), \delta)$-differentially private measurement is also $(\texttt{epsilon}, \delta)$-differentially private.
\end{proof}


\bibliographystyle{alpha}
\bibliography{mod}

\end{document}